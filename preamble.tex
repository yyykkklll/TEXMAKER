%% This file contains a number of tweaks that are typically applied to the main document.
%% They are not enabled by default, but can be enabled by uncommenting the relevant lines.

%%
%% Inline annotations; for predefined colors, refer to "dvipsnames" in the xcolor package:
%% https://tinyurl.com/overleaf-colors
%%
\newcommand{\red}[1]{{\color{red}#1}}
\newcommand{\todo}[1]{{\color{red}#1}}
\newcommand{\TODO}[1]{\textbf{\color{red}[TODO: #1]}}
%%
%% disable for camera ready / submission by uncommenting these lines  
%%
% \renewcommand{\TODO}[1]{}
% \renewcommand{\todo}[1]{#1}

%%
%% work harder in optimizing text layout. Typically shrinks text by 1/6 of page, enable
%% it at the very end of the writing process, when you are just above the page limit
%%
% \usepackage{microtype}

%%
%% fine-tune paragraph spacing
%%
% \renewcommand{\paragraph}[1]{\vspace{.5em}\noindent\textbf{#1.}}

%%
%% globally adjusts space between figure and caption
%%
% \setlength{\abovecaptionskip}{.5em}


%%
%% Allows "the use of \paper to refer to the project name"
%% with automatic management of space at the end of the word
%%
% \usepackage{xspace}
% \newcommand{\paper}{ProjectName\xspace}

%%
%% Commonly used math definitions
%%
% \DeclareMathOperator*{\argmin}{arg\,min}
% \DeclareMathOperator*{\argmax}{arg\,max}

%%
%% Tigthen underline
%%
% \usepackage{soul}
% \setuldepth{foobar}

%% ========================================
%% Additional packages for the paper
%% ========================================
\usepackage{graphicx}
\usepackage{amsmath}
\usepackage{amssymb}
\usepackage{booktabs}
\usepackage{tabularx}
\usepackage{multirow}
\usepackage{float}
\usepackage{hhline}
\usepackage{caption}
\usepackage[activate={true,nocompatibility},final,tracking=true,kerning=true,stretch=10,shrink=10]{microtype}
\usepackage{tabularray}
\usepackage[table]{xcolor}
\usepackage{colortbl}
\usepackage{stfloats}

% Page layout settings
\setlength{\parskip}{0pt}
\setlength{\parindent}{2em}
\linespread{1.05}

% Custom commands
\newcommand{\thickhline}{%
    \noalign{\global\dimen1=\arrayrulewidth\global\arrayrulewidth=1pt}%
    \hline
    \noalign{\global\arrayrulewidth=\dimen1}%
}
\newcommand{\thickhhline}[1]{%
    \noalign{\global\dimen1=\arrayrulewidth\global\arrayrulewidth=1pt}%
    \hhline{#1}%
    \noalign{\global\arrayrulewidth=\dimen1}%
}


